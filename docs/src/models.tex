\subsection{\bitt{model}\label{model}}

Declare a model.

\subsubsection*{Synopsis\label{model_Synopsis}}
\begin{bicode}
model \textsl{Name} \{
  \(\ldots\)
\}
\end{bicode}

\subsubsection*{Description\label{model_Description}}

A \bitt{model} statement declares and names a model, and encloses
declarations of the constants, dimensions, variables, inlines and top-level
blocks that specify that model.

The following named top-level blocks are supported, and should usually be
provided:
\begin{itemize}
\item \blockref{parameter}, specifying the prior density over parameters,
\item \blockref{initial}, specifying the prior density over initial conditions,
\item \blockref{transition}, specifying the transition density, and
\item \blockref{observation}, specifying the observation density.
\end{itemize}

The following named top-level blocks are supported, and may optionally be
provided:
\begin{itemize}
\item \blockref{proposal_parameter}, specifying a proposal density over parameters,
\item \blockref{proposal_initial}, specifying a proposal density over initial conditions,
\item \blockref{lookahead_transition}, specifying a lookahead density to accompany the transition density, and
\item \blockref{lookahead_observation}, specifying a lookahead density to accompany the observation density.
\end{itemize}

\subsection{\bitt{dim}\label{dim}\index{dim}}

Declare a dimension.

\subsubsection*{Synopsis\label{dim_Synopsis}}
\begin{bicode}
dim \textsl{name}(100, 'cyclic')
dim \textsl{name}(size = 100, boundary = 'cyclic')
\end{bicode}

\subsubsection*{Description\label{dim_Description}}

A \bitt{dim} statement declares a dimension with a given size and boundary
condition.

A dimension\index{dimension} may be declared anywhere in a model
specification. Its scope is restricted to the block in which it is
declared\index{scope}. A dimension must be declared before any variables that
extend along it are declared.

\subsubsection*{Arguments\label{dim_Arguments}}

\begin{description}
\item[\bitt{size}] (position 0, mandatory)

Length of the dimension.

\item[\bitt{boundary}] (position 1, default 'none')

Boundary condition of the dimension. Valid values are:

\begin{description}
\item[\bitt{'none'}]

No boundary condition.

\item[\bitt{'cyclic'}]

Cyclic boundary condition; all indexing is taken modulo the \bitt{size} of
the dimension.
\end{description}
\end{description}

\subsection{\bitt{input}\index{input}\label{input}, \bitt{noise}\index{noise}\label{noise}, \bitt{obs}\index{obs}\label{obs}, \bitt{param}\index{param}\label{param} and \bitt{state}\index{state}\label{state}}

Declare an input, noise, observed, parameter or state variable.

\subsubsection*{Synopsis\label{var_Synopsis}}

\begin{bicode}
state x                       // scalar variable
state x[i]                    // vector variable
state X[i,j]                  // matrix variable
state X[i,j,k]                // higher-dimensional variable
state X[i,j](has_output = 0)  // omit from output files
state x, y, z                 // multiple variables
\end{bicode}

\subsubsection*{Description\label{var_Description}}

Declares a variable of the given type, extending along the dimensions listed
between the square brackets.

A variable\index{variable} may be declared anywhere in a model
specification. Its scope is restricted to the block in which it is
declared\index{scope}. Dimensions along which a variable extends must be
declared prior to the declaration of the variable, using the \kwref{dim}
statement.

\subsubsection*{Arguments\label{var_Arguments}}

\begin{description}
\item[\bitt{has\_input}] (default 1)\index{I/O}

Include variable when doing input from a file?

\item[\bitt{has\_output}] (default 1)\index{I/O}

Include variable when doing output to a file?

\item[\bitt{input\_name}] (default the same as the name of the
  variable)\index{I/O}

Name to use for the variable in input files.

\item[\bitt{output\_name}] (default the same as the name of the
  variable)\index{I/O}

Name to use for the variable in output files.

\end{description}

\subsection{\bitt{const}\label{const}\index{const}}

Declare a constant\index{constant}.

\subsubsection*{Synopsis\label{const_Synopsis}}
\begin{bicode}
const \textsl{name} = \textsl{constant_expression}
\end{bicode}

\subsubsection*{Description\label{const_description}}

A \bitt{const}\index{const} statement declares a constant\index{constant}, the
value of which is evaluated using the given constant
expression\index{constant\,expression}. The constant may then be used, by
name, in other expressions.

A constant\index{constant} may be declared anywhere in a model
specification. Its scope is restricted to the block in which it is
declared\index{scope}.

\subsection{\bitt{inline}\label{inline}\index{inline}}

Declare an inline\index{inline expression}.

\subsubsection*{Synopsis\label{inline_synopsis}}
\begin{bicode}
inline \textsl{name} = \textsl{expression}
...
x <- 2*\textsl{name}  // equivalent to x <- 2*(\textsl{expression})
\end{bicode}

\subsubsection*{Description\label{inline_description}}

An \bitt{inline}\index{inline} statement declares an inline\index{inline
  expression}, the value of which is an expression that will be substituted in
place of any occurrence of the inline's name in other expressions. The inline
may be used in any expression where it will not violate the constraints on
that expression (e.g. an inline expression that refers to a \kwref{state}
variable may not be used within a constant expression).

An inline expression\index{inline expression} may be declared anywhere in a
model specification. Its scope is restricted to the block in which it is
declared\index{scope}.
