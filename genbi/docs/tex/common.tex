\subsection{Common options\label{Common_options}}

The options in this section are common across all client programs.

Options prefixed with {\sf --enable} may be negated by replacing this with
{\sf --disable}. Likewise, options prefixed with {\sf --with} may be negated
by replacing this with {\sf --without}.

\subsubsection{General options}
\begin{itemize}
\item {\sf --help}

Display help.

\item {\sf --verbose} (default off)

Enable verbose reporting.
\end{itemize}

\subsubsection{Build options}

\begin{itemize}
\item {\sf --dry-build} (default off)

Do not build.

\item {\sf --warn} (default off)

Enable compiler warnings.

\item {\sf --native} (default off)

Compile for native platform.

\item {\sf --force} (default off)

Force all build steps to be performed, even when determined to be not
required. This is useful as a workaround of any faults in detecting
changes, or when system headers or libraries change and recompilation or
relinking is required.

\item {\sf --enable-assert} (default on)

Enable assertion checking (recommended for test runs, not recommended for
production runs).

\item {\sf --enable-cuda} (default off)

Enable CUDA device code.

\item {\sf --enable-sse} (default off)

Enable SSE host code.

\item {\sf --enable-mkl} (default off)

Enable Intel MKL code.

\item {\sf --enable-mpi} (default off)

Enable MPI code.

\item {\sf --enable-vampir} (default off)

Enable Vampir.

\item {\sf --enable-single} (default off)

Use single-precision floating point.
\end{itemize}

\subsubsection{Run options}
\begin{itemize}
\item {\sf --dry-run}

Do not run.

\item {\sf --seed} (default 0)

Pseudorandom number generator seed.

\item {\sf --threads \textsl{N}} (default 0)

Run with {\sf \textsl{N}} threads. If zero, the number of threads used is the
default for OpenMP on the platform.

\item {\sf --enable-timing} (default off)

Enable timings.

\item {\sf --enable-output} (default on)

Enable output.

\item {\sf --with-gdb} (default off)

Run within the {\sf gdb} debugger.

\item {\sf --with-cuda-gdb} (default off)

Run within the {\sf cuda-gdb} debugger.

\item {\sf --with-valgrind} (default off)

Run within {\sf valgrind}.

\item {\sf --with-gperftools} (default off)

Run with {\sf gperftools} profiler.

\item {\sf --gperftools-file} (default '\textsl{command}.prof')

Output file to use under {\sf --with-gperftools}.

\item {\sf --with-mpi} (default off)

Run with {\sf mpirun}.

\item {\sf --mpi-np}

Number of processes when running under {\sf mpirun}.

\end{itemize}

\subsubsection{Model transformation options}
\begin{itemize}
\item {\sf --transform-extended} (default off, enabled automatically when
  required)

Linearise the model and symbolically compute Jacobian expressions for use with
the extended Kalman filter.

\item {\sf --transform-param-to-state} (default off)

Augment the state with parameters by interpreting {\sf param} statements as
{\sf state}, merging the {\sf parameter} top-level block into the {\sf
  initial} top-level block, and merging the {\sf proposal\_parameter}
top-level block into the {\sf proposal\_initial} top-level block.

This is useful for joint state and parameter estimation using filters.

\item {\sf --transform-initial-to-param} (default off)

Augment the parameters with the initial conditions of the state variables by
adding a new {\sf param} for each {\sf state} variable to hold its initial
value, merging the {\sf initial} top-level block into the {\sf parameter}
top-level block, merging the {\sf proposal\_initial} top-level block into the
{\sf proposal\_parameter} top-level block, and replacing both the {\sf
  proposal\_parameter} and {\sf proposal\_initial} top-level blocks with
$\delta$-masses.

This is useful for treating the initial conditions as parameters in sampling
schemes such as PMCMC and SMC$^2$. Note that the transformation is semantic,
not substantial.

\end{itemize}
