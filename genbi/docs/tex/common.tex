\subsection{Common options\label{Common_options}}

The options in this section are common across all client programs.

\subsubsection{General options}
\begin{itemize}
\item {\sf --help} Display help.
\item {\sf --verbose} Enable verbose reporting.
\end{itemize}

\subsubsection{Build options}
\begin{itemize}
\item {\sf --dry-build} Do not build.
\item {\sf --warn} Enable compiler warnings.
\item {\sf --debug} Enable assertion checking.\index{assertion checking}
\item {\sf --profile} Enable profiling with {\sf gprof}.\index{profiling}\index{gprof}
\item {\sf --native} Compile for native platform.
\item {\sf --gpu} Enable CUDA device code.\index{GPU}\index{CUDA}
\item {\sf --sse} Enable SSE host code.\index{SSE}\index{SIMD}
\item {\sf --double} Use double-precision floating point.
\item {\sf --force} Force all build steps to be performed, even when
  determined not to be required. This is useful as a workaround of any faults
  in detecting changes in dependencies, or when system headers or libraries
  change and recompilation or relinking is required.
\end{itemize}

\subsubsection{Run options}
\begin{itemize}
\item {\sf --dry-run} Do not run.
\item {\sf --threads \textit{N}} Run with {\sf \textit{N}} threads\index{multithreading}.
\item {\sf --gdb} Run within {\sf gdb}\index{gdb}.
\item {\sf --cuda-gdb} Run within {\sf cuda-gdb}\index{cuda-gdb}.
\item {\sf --valgrind} Run within {\sf valgrind}\index{valgrind}.
\end{itemize}
