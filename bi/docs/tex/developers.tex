%%%%
\section{Obtaining the source code}

Bi is available via Subversion at
\hyperref[hyper][https://svnserv.csiro.au/svn/css/modelling/bgc/code/bi]{https://svnserv.csiro.au/svn/css/modelling/bgc/code/bi}. You
would typically check out a working copy of the \textsf{trunk} directory:
\begin{alltt}{\sf
svn co https://svnserv.csiro.au/svn/css/modelling/bgc/code/bi/trunk bi
}\end{alltt}

For any endeavours that may temporarily break functionality, working on a
branch, rather than the main trunk, is essential. A branch can be created by
copying the trunk into an appropriately named subdirectory of the
\textsf{branches} directory:
\begin{alltt}{\sf
svn copy -m 'Created branch for...' \textbackslash
  https://svnserv.csiro.au/svn/css/modelling/bgc/code/bi/trunk \textbackslash
  https://svnserv.csiro.au/svn/css/modelling/bgc/code/bi/branches/branch_name
}\end{alltt}
and then checking out a working copy of the branch instead. Branches are not
required to actually work at any time, so regular commits are encouraged, even
for code that does not yet compile or run correctly. Note that you may like to
copy only a particular subdirectory of the trunk, rather than the entire
thing.

Once modifications are complete, a branch can be merged back into a working
copy of the trunk:
\begin{alltt}{\sf
svn merge --reintegrate \textbackslash
  https://svnserv.csiro.au/svn/css/modelling/bgc/code/bi/branches/branch_name
}\end{alltt}
and then the working copy of the trunk committed.

\section{Using the Eclipse IDE}

Bi is set up to use the Eclipse IDE. LibBi works with the Eclipse IDE for
C/C++, and GenBi with the EPIC (Eclipse Perl Integration) project. From a
vanilla install of the Eclipse IDE for C/C++ (available at
\hyperref[hyper][http://www.eclipse.org]{http://www.eclipse.org}), a number of
additional plugins are supported, and recommended. These are, along with the
URLs of their update sites:
\begin{itemize}
\item Subclipse (Subversion integration)
  \hyperref[hyper][http://subclipse.tigris.org/update]{http://subclipse.tigris.org/update\_1.8.x}
  (change 1.8 to match your Subversion version)

\item Eclox (Doxygen integration) \hyperref[hyper][http://download.gna.org/eclox/update]{http://download.gna.org/eclox/update}

\item EPIC (Perl integration) \hyperref[hyper][http://e-p-i-c.sf.net/updates/testing]{http://e-p-i-c.sf.net/updates/testing}

\item Perl Template Toolkit Editor \hyperref[hyper][http://perleclipse.com/TTEditor/UpdateSite]{http://perleclipse.com/TTEditor/UpdateSite}
\end{itemize}
They may be installed via the \textsf{Help > Install New Software...} menu
item in Eclipse, in each case entering the update site URL.

With Subclipse installed, each of LibBi, GenBi and OctBi may be checked out of
the Subversion repository as a separate project in Eclipse. Use the
\textsf{File > New > Other...} menu item and select the \textsf{SVN > Checkout
  Projects from SVN} item from the list. Enter the URL of the Bi Subversion
repository trunk (or of a branch), then select one or more of the
\textsf{libbi}, \textsf{genbi} or \textsf{octbi} directories to check these
out as Eclipse projects.

For LibBi, you will need to run \textsf{./autogen.sh} then
\textsf{./configure} from the command line before you can build from within
Eclipse.

A custom code style is available in the \textsf{custom\_code\_style.xml} file,
and should be used. This may be loaded via the \textsf{Window > Preferences}
dialog, using the \textsf{Import...} button under \textsf{C/C++ > Code Style}.


%%%%
\section{Developing GenBi}

Developing Bi involves adding or modifying functionality in one or more of its
major \hyperref[hyper][Components]{components}. A typical exercise is adding
block and action types to GenBi to introduce new features, such as support for
additional probability density functions. This may also involve implementing
new functionality in LibBi on which these blocks and actions can draw. Another
common task is adding a new inference method to LibBi, then adding or
modifying templates in GenBi to produce the client code that will use this.

\subsection{Actions, blocks and clients}

A user of Bi is exposed only to those blocks and actions which they explicitly
write in their model specification file. Developers must be aware that beyond
these \textit{explicit} blocks, additional \textit{implicit} blocks are
inserted according to the actions specified by the user. In particular, each
action has a preferred parent block type. If the user, in writing their model
specification, does not explicitly place an action within its
preferred block type, the action will be implicitly wrapped in it anyway.

For example, take the model specification:
\begin{alltt}{\sf
    sub transition \{
      do \{
        x <- a*x + b
        y <- c*y + d
      \} then \{
        x ~ uniform(x - 1.0, x + 1.0)
        y ~ gaussian(y, 1.0)
      \}
    \}
}\end{alltt}
The \hyperref[hyper][uniform]{\sf uniform}\index{uniform} and
\hyperref[hyper][gaussian]{\sf gaussian}\index{gaussian} actions prefer
different parent block types. These additional block types will be inserted,
to give the equivalent of the user having specified:
\begin{alltt}{\sf
    sub transition \{
      do \{
        x <- a*x + b
        y <- c*y + d
      \} then \{
        sub uniform_ \{
          x ~ uniform(x - 1.0, x + 1.0)
        \}
        sub gaussian_ \{
          y ~ gaussian(y, 1.0)
        \}
      \}  
    \}
}\end{alltt}

For the developer, then, actions and blocks are always written in tandem.
Note the underscore suffix on \textsf{uniform\_} and \textsf{gaussian\_} is a
convention that marks these as blocks for internal use only -- the user is not
intended to use them explicitly (see the \hyperref[hyper][Style_guide]{\sf
  Style guide}\index{style guide}).

It is worth familiarising yourself with this behaviour, and indeed other
transformations made by GenBi, by using the
\hyperref[hyper][rewrite]{\sf rewrite}\index{rewrite} command.

\subsubsection{Adding a block}

To add a block:
\begin{enumerate}
\item Choose a name for the block.
\item Create a Perl module \textsf{Bi::Block::\textit{name}} in the
  file \textsf{gen/src/Bi/Block/\textit{name}.pm}. Note that Perl module
  ``CamelCase'' naming conventions should be ignored in favour of Bi block
  naming conventions here.
\item Create a new Perl Template Toolkit template
  \textsf{gen/src/tt/block/\textit{name}.hpp.tt}. When rendered, the template
  will be passed a single variable named \textsf{block}, which is an object of
  the class created in the previous step.
\end{enumerate}

As implementation details of the Perl module and template are subject to
change, it is highly recommended that you copy and modify an existing, similar
block type, as the basis for your new block type.

\subsubsection{Adding an action}

To add an action:
\begin{enumerate}
\item Choose a name for the action.
\item Create a Perl module \textsf{Bi::Block::\textit{name}} in the
  file \textsf{gen/src/Bi/Block/\textit{name}.pm}. Note that Perl module
  ``CamelCase'' naming conventions should be ignored in favour of Bi action
  naming conventions here.
\item If necessary, create a new Perl Template Toolkit template
  \textsf{gen/src/tt/action/\textit{name}.hpp.tt}. When rendered, the template
  will be passed a single variable named \textsf{action}, which is an object
  of the class created in the previous step. Often, this template is not
  necessary, such as when the template for the containing block generates all
  the code required.
\end{enumerate}

\subsubsection{Adding a client\label{Adding_a_client}}

To add a client:
\begin{enumerate}
\item Consider whether an existing client can be modified or a new client
  should be added. For an existing client, edit the corresponding Perl module
  \textsf{src/Bi/Client/\textit{name}.pm}. For a new client, choose a name
  and create a new Perl module \textsf{src/Bi/Client/\textit{name}.pm}. Again,
  Bi client naming conventions should be favoured over Perl module naming
  conventions.
\item Consider whether an existing template can be modified or a new template
  should be created. For an existing template, edit the corresponding template
  in \textsf{tt/cpp/client/\textit{template}.cpp.tt}. For a new template:
\begin{enumerate}
\item Create files \textsf{tt/cpp/client/\textit{template}.cpp.tt} and
  \textsf{tt/cpp/client/\textit{template}.cu.tt}. Note that the second
  typically only needs to \textsf{\#include} the former, the \textsf{*.cu}
  file extension is merely needed for a CUDA-enabled compile.
\item Modify the Perl module to select this template for use where
  appropriate.
\item Add the template name to the list given at the top of
  \textsf{tt/build/Makefile.am.tt} to ensure that a build fule is generated
  for it.
\end{enumerate}
\end{enumerate}

\subsection{Designing an extension\label{Designing_an_extension}}

Given that there are no particular requirements on actions other than those
imposed by your own containing block class, there is considerable flexibility
in designing block and action behaviour. This section provides some guidance
as to best practice.

It is recommended that most logic is written in blocks, rather than
actions. This allows multiple actions to be combined where doing so will
improve performance (e.g. minimising the number of kernel launches on a GPU
device, and maximising parallelism). Note that, unless otherwise specified,
arguments to an action are unrolled and evaluated prior to any calls to the
action and its containing block. All arguments are thus passed as the
variables holding the evaluated result.

Complex C++ code should form part of LibBi rather than GenBi. The C++ code
generated by templates should typically be limited to arranging for a few
function calls into LibBi, where most of the work will be done.

When writing templates for client programs, consider that it is advantageous
for the user that they can change command line arguments without the C++ code
changing, and thus triggering a recompile. This can be achieved by writing
runtime checks on command-line options in C++ rather than code-generation-time
checks in the Perl Template Toolkit. This need not be taken to the extreme,
however: clear and simple template code is preferred over convoluted C++!

\subsection{Documenting an extension\label{Documenting_an_extension}}

Reference documentation for actions, blocks and clients is written in the Perl
module created for them, in standard Perl POD. Likewise, parameters for
actions and blocks, and command-line options for clients, are enumerated in
these. The idea of this is to keep implementation and documentation together.

POD documentation in these modules is automatically converted to LaTeX and
incorporated in the \hyperref[hyper][Reference]{Reference} section of this
manual.

\subsection{Style guide\label{Developer_style_guide}}\index{style guide}

Further to the \hyperref[hyper][Style_guide]{Style guide} for users, the
following additional recommended pertain to developers:
\begin{itemize}
\item Action, block, client and argument names are all lowercase, with
  multiple words separated by '\_' (the underscore). Uppercase may be used in
  exceptional cases where this convention becomes contrived. A good example is
  matrix arguments, which might naturally be named with uppercase letters.
\item Actions, blocks and arguments that are not meant to be used explicitly
  should be suffixed with a single underscore.
\end{itemize}

\section{Developing LibBi}

\subsection{Header files}

Header files are given two extensions in LibBi:
\begin{enumerate}
\item {\sf *.hpp} headers may be safely included in any C++ (e.g.
{\sf *.cpp}) or CUDA ({\sf *.cu}) source files,
\item {\sf *.cuh} headers may only be safely included in CUDA ({\sf *.cu})
source files. They include CUDA C extensions.
\end{enumerate}

Note that not all {\sf *.hpp} headers can be safely included in {\sf *.cu}
files either, due to CUDA compiler limitations, particularly those headers
that further include Boost headers which make extensive use of
templates. Efforts have been made to quarantine such incidences from the CUDA
compiler, but mileage may vary.

\subsection{Coding conventions}

The following names are used for template parameters (where {\sf n}>
is an integer):
\begin{itemize}
\item {\sf B} for the model type,
\item {\sf Tn} for scalar types,
\item {\sf Vn} for matrix types,
\item {\sf Mn} for vector types,
\item {\sf Qn} for pdf types,
\item {\sf S} for type lists,
\item {\sf L} for location (host or device),
\item {\sf CL} for location of a cache,
\item {\sf PX} for parents type,
\item {\sf CX} for coordinates type,
\item {\sf OX} for output type.
\end{itemize}

\subsection{Pseudorandom reproducibility\label{Pseudorandom_reproducibility}}\index{pseudorandom numbers}

It is important to maintain reproducibility of results under the same random
number seed\index{random number seed}, typically passed using the
\textsf{--seed} option on the command line. Real time impacts should be
considered, such as favouring static scheduling\index{static scheduling} over
dynamic scheduling\index{dynamic scheduling} for OpenMP\index{OpenMP} thread
blocks. Consider the following:
\begin{alltt}{\sf
    Random rng;

    #pragma omp parallel for
    for (int i = 0; i < N; ++i) \{
       x = rng.uniform();
       \(\ldots\)
    \}
}\end{alltt}

\textsf{Random} maintains a separate pseudorandom number
generator\index{pseudorandom number generation} for each OpenMP thread. If
dynamically scheduled, the above loop gives no guarantees as to the number of
variates drawn from each generator, so that reproducibility of results for a
given seed is not guaranteed. Static scheduling should be enforced in this
case to ensure reproducibility:
\begin{alltt}{\sf
    Random rng;

    #pragma omp parallel for schedule(static)
    for (int i = 0; i < N; ++i) \{
       x = rng.uniform();
       \(\ldots\)
    \}
}\end{alltt}

A more subtle consideration is the conditional generation of
variates. Consider the following code, evaluating a
Metropolis-Hastings\index{Metropolis-Hastings} acceptance criterion:
\begin{alltt}{\sf
    alpha = (l1*p1*q2)/(l2*p2*q1);
    if (alpha >= 1.0 || rng.uniform() < alpha) \{
      accept();
    \} else \{
      reject();
    \}
}\end{alltt}
Here, \textsf{rng.uniform()} is called only when \textsf{alpha >= 1.0} (given
operator short-circuiting). We might prefer, however, that across multiple
runs with the same seed, the pseudorandom number generator is always in the
same state for the $n$th iteration, regardless of the acceptance criteria
across the preceding iterations. Moving the variate generation outside the
conditional will fix this:
\begin{alltt}{\sf
    alpha = (l1*p1*q2)/(l2*p2*q1);
    u = rng.uniform();
    if (alpha >= 1.0 || u < alpha) \{
      accept();
    \} else \{
      reject();
    \}
}\end{alltt}

\subsection{Shallow copy, deep assignment}

Typically, classes in LibBi follow a ``shallow copy, deep assignment''
idiom. The idiom is most apparent in the case of vector and matrix types. A
copy of a vector or matrix object is shallow -- the new object merely contains
a pointer to the memory buffer underlying the existing object. An assignment
of a vector or matrix is deep, however -- the contents of the memory buffer
underlying the existing object is copied into that of the new object.

Often the default copy constructor is sufficient to achieve a shallow copy,
while an override of the default assignment operator may be necessary to
achieve deep assignment.

\emph{Generic} copy constructors and assignment operators are also
common. These are templated overloads of the default versions which accept, as
arguments, objects of some class other than the containing class. When a
generic copy constructor or assignment operator is used, the default copy
constructor or assignment operator should always be overridden also.
