\section{Introduction\label{Introduction}}

Bi (short for ``Bayesian inference'') is used for probabilistic inference over
state-space models, including simulation, filtering and smoothing for state
estimation, and optimisation and sampling for parameter estimation.

Bi supports state-space models of the form:
\begin{equation*}
p(\mathbf{y}_{1:T},\mathbf{x}_{1:T},\boldsymbol{\theta}) =
\underbrace{p(\boldsymbol{\theta})}_{\text{parameter}}
\underbrace{p(\mathbf{x}_0|\boldsymbol{\theta})}_{\text{initial}}
\prod_{t=1}^T
\underbrace{p(\mathbf{x}_t|\mathbf{x}_{t-1},\boldsymbol{\theta})}_{\text{transition}}
\underbrace{p(\mathbf{y}_t|\mathbf{x}_t,\boldsymbol{\theta})}_{\text{observation}}.
\end{equation*}
where $t = 1,\ldots,T$ indexes time, $\mathbf{y}_{1:T}$ are observations,
$\mathbf{x}_{1:T}$ are state variables, and $\boldsymbol{\theta}$ are
parameters.

The model is hierarchically factored into four particular conditional
probability densities:
\begin{itemize}
\item the parameter model, specifying the prior density over parameters,
\item the initial condition model, specifying the prior density over initial
  conditions, conditioned on the parameters,
\item the transition model, specifying the transition density conditioned on
  the parameters and previous state,
\item the observation model, specifying the observation density conditioned on
  the parameters and current state.
\end{itemize}
Each of these is explicitly specified using the
\hyperref[hyper][Modelling_language]{Bi modelling language}\index{modelling
  language}.

We begin with a brief example to set the scene. Consider the following
Lotka-Volterra-like predator-prey model between zooplankton (predator, $Z$)
and phytoplankton (prey, $P$):
\begin{eqnarray*}
\frac{dP}{dt} &=& \alpha_t P - cPZ\\
\frac{dZ}{dt} &=& ecPZ - m_lZ - m_q Z^2.
\end{eqnarray*}
Here, $t$ is time (in days), with prescribed constants $c = .25$, $e = .3$,
$m_l = .1$ and $m_q = .1$. The stochastic growth term, $\alpha_t$, is updated
in discrete time by drawing $\alpha_t \sim \mathcal{N}(\mu,\sigma)$
daily. Parameters to be estimated are $\mu$ and $\sigma$, and $P$ is observed,
with noise, at daily intervals.

The model above might be specified in the Bi modelling language as follows:
\begin{alltt}{\sf
    /**
     * Lotka-Volterra-like phytoplankton-zooplankton (PZ) model.
     */
    model PZ \{
      const c = 0.25     // zooplankton clearance rate
      const e = 0.3      // zooplankton growth efficiency
      const m_l = 0.1    // zooplankton linear mortality
      const m_q = 0.1    // zooplankton quadratic mortality

      param mu, sigma     // mean and standard deviation of phytoplankton growth
      state P, Z         // phytoplankton, zooplankton
      noise alpha        // stochastic phytoplankton growth rate
      obs P_obs          // observations of phytoplankton
  
      sub parameter \{
        mu ~ uniform(0.0, 1.0)
        sigma ~ uniform(0.0, 0.5)
      \}
  
      sub initial \{
        P ~ log_normal(log(2.0), 0.2)
        Z ~ log_normal(log(2.0), 0.1)
      \}

      sub transition(delta = 1.0) \{
        do \{
          alpha ~ normal(mu, sigma)
        \} then ode(atoler = 1.0e-3, rtoler = 1.0e-3, alg = 'rk4') \{
          P <- ode(alpha*P - c*P*Z)
          Z <- ode(e*c*P*Z - m_l*Z - m_q*Z*Z)
        \}
      \}

      sub observation \{
        P_obs ~ log_normal(log(P), 0.2)
      \}
    \}
}\end{alltt}

This would be saved in a file named {\sf PZ.bi}. We can now consider
performing various tasks with the Bi command-line interface, the simplest of
which is simulating from the model:
\begin{alltt}{\sf
bi simulate --model-file PZ.bi -P 128 -T 365 -K 365 --output-file results/simulate.nc
}\end{alltt}
This command will simulate 128 trajectories of the model ({\sf -P 128}), each
of 365 days ({\sf -T 365}), outputting the results to the NetCDF\index{NetCDF}
file {\sf results/simulate.nc} every day ({\sf -K 365}).

The \hyperref[hyper][Command_line_interface]{command-line interface} provides
numerous other functionality, including filtering and smoothing the model with
respect to data, and optimising or sampling its parameters.

\section{Components\label{Components}}

Bi consists of three components:
\begin{description}
\item[LibBi,]\index{LibBi} which provides classes and functions for simulation
  and inference, numerical computing, memory management and I/O. It is written
  in C++ with extensive use of generic programming concepts.
\item[GenBi,]\index{GenBi} which parses the Bi modelling language, constructs
  and optimises an internal model representation, and generates C++ code for
  compiling against LibBi. It is written in Perl using object-oriented
  concepts, and the Perl Template Toolkit for code generation.
\item[OctBi,]\index{OctBi} which provides a set of functions for GNU Octave
  for querying, analysing and visualising outputs.
\end{description}

The \hyperref[hyper][Command_line_interface]{command-line interface} provides
for interaction with LibBi and GenBi. OctBi can be installed separately as a
GNU Octave package for easily working with output files from this.

\section{The Bi modelling language\label{Modelling_language}}

A model specification file contains a single model specification. The
specification always starts with an outer \hyperref[hyper][model]{\sf
  model}\index{model} statement that declares and names the model. It then
proceeds with declarations of dimensions and variables, followed by four
\emph{top-level blocks} -- \hyperref[hyper][parameter]{\sf parameter},
\hyperref[hyper][initial]{\sf initial}, \hyperref[hyper][transition]{\sf
  transition} and \hyperref[hyper][observation]{\sf observation} -- that
describe the factors of the hierarchical model.

A suitable template is:
\begin{alltt}{\sf
    model \textsl{Name} \{
      // declare dimensions...
      // declare variables...

      sub parameter \{
        // specify the parameter model
      \}

      sub initial \{
        // specify the initial condition model
      \}

      sub transition \{
        // specify the transition model
      \}

      sub observation \{
        // specify the observation model
      \}
    \}
}\end{alltt}

Note that the contents of the \hyperref[hyper][model]{\sf model}\index{model}
statement and each top-level block are contained in curly braces ({\sf
\{\(\ldots\)\}}), in typical C-style. Comments are also C-style, an inline
comment being wrapped by \textsf{/*} and \textsf{*/}, and the double-slash
(\textsf{//}) denoting an end-of-line comment. Semicolons at the end of each
line are optional.

Each top-level blocks may contain further nested \emph{blocks}, which
in turn contain \emph{actions}, each of which specifies the distribution of,
or means to evaluate, a variable.

\subsection{Dimensions\label{Dimensions}}

\subsection{Variables\label{Variables}}

\subsection{Actions\label{Actions}}

Within each top-level block, a probability density is specified using
\emph{actions}. If \textsf{x} is some variable already declared, an action
takes the form:
\begin{alltt}{\sf
    x ~ \textsl{name}(\textsl{arguments}, ...)
}\end{alltt}
or
\begin{alltt}{\sf
    x <- \textsl{name}(\textsl{arguments}, ...)
}\end{alltt}

The first form indicates that the (random) variable \textsf{x} is distributed
according to the action given on the right. Such actions are usually, and
unsurprisingly, named after common parametric distributions (e.g.
\hyperref[hyper][gaussian]{\sf gaussian}\index{gaussian},
\hyperref[hyper][gamma]{\sf gamma}\index{gamma} and
\hyperref[hyper][uniform]{\sf uniform}\index{uniform}).

The second form indicates that the (deterministic) variable \textsf{x} should
be assigned a value evaluated according to the action given on the right. Such
actions may require the evaluation of a simple scalar, vector or matrix
expression, or may required more complicated processing, such as the numerical
integration of ordinary differential equations\index{ordinary differential
  equations} (e.g. the \hyperref[hyper][ode]{\sf ode}\index{ode} action).

Actions support both \emph{named}\index{named arguments} and \emph{positional}
arguments\index{positional arguments}, and this will be indicated in their
documentation. Positional arguments are interpreted by the order given. For
example
\begin{alltt}{\sf
    x ~ gaussian(0.0, 1.0)
}\end{alltt}
means that \textsf{x} is distributed according to the
\hyperref[hyper][gaussian]{\sf gaussian}\index{gaussian} action with, by the
definition of that action, mean $0.0$ and standard deviation $1.0$.

The \hyperref[hyper][gaussian]{\sf gaussian}\index{gaussian} action also
happens to support named arguments, so the following is equivalent:
\begin{alltt}{\sf
    x ~ gaussian(mean = 0.0, std = 1.0)
}\end{alltt}
The order of named arguments is unimportant, so the following is also
equivalent:
\begin{alltt}{\sf
    x ~ gaussian(std = 1.0, mean = 0.0)
}\end{alltt}

Positional and named arguments may be mixed, but all positional arguments must
appear before all named arguments. Thus, this is valid:
\begin{alltt}{\sf
    x ~ gaussian(0.0, std = 1.0)
}\end{alltt}
but these are not:
\begin{alltt}{\sf
    x ~ gaussian(mean = 0.0, 1.0)
    x ~ gaussian(std = 1.0, 0.0)
}\end{alltt}
This avoids ambiguity.

Refer to the definition of a particular action to determine which arguments
may be passed in positional, named or either form.

\subsection{Serialisation\label{Serialisation}}

It is assumed that all actions within the same block are independent, and can
be evaluated concurrently. Where dependencies exist between actions, the
\hyperref[hyper][do_then]{\sf do.. then}\index{do.. then} clause may be used to
serialise actions so as to satisfy prerequisites:
\begin{alltt}{\sf
    do \{
      eps ~ gaussian(0.0, 1.0)
    \} then \{
      x <- A*x + eps
    \}
}\end{alltt}
Here, the variable \textsf{x} depends on the variable \textsf{eps} first being
evaluated.

Any number of {\sf then} parts may follow the initial {\sf do}.

Under the hood, Bi assumes that it can evaluate actions within the same block
in parallel, unless it is told otherwise. The \hyperref[hyper][do_then]{\sf
  do.. then}\index{do.. then} clause is the only way to tell it otherwise.

\subsection{Blocks}

Some actions are more complicated and can be wrapped in \emph{named blocks}. A
good example is the \hyperref[hyper][ode]{\sf ode}\index{ode} action. This
describes the differential equation\index{ordinary differential equations} for
a single variable, which must be grouped with other actions that describe the
differential equations of other variables in the same system. The grouping is
achieved by the \hyperref[hyper][ode]{\sf ode}\index{ode} named block.

Consider the following Lotka-Volterra-type\index{Lotka-Volterra} transition
model:
\begin{alltt}{\sf
    sub transition \{
      sub ode \{
        P <- ode(alpha*P - c*P*Z)
        Z <- ode(e*c*P*Z - m_l*Z)
      \}
    \}
}\end{alltt}
Here, the additional \hyperref[hyper][ode]{\sf ode}\index{ode} block combines
the two actions into the one system of ordinary differential equations.

Note that every action is automatically wrapped in an appropriate enclosing
block like this by GenBi. Indeed, in this case, GenBi will implicitly wrap the
two \hyperref[hyper][ode]{\sf ode}\index{ode} actions in the one
\hyperref[hyper][ode]{\sf ode}\index{ode} block anyway, so that explicitly
doing so is unnecessary (but unharmful). Like actions, however, blocks can
take arguments; explicitly giving the enclosing block permits arguments to be
changed. In this case, the \hyperref[hyper][ode]{\sf ode}\index{ode} block has
parameters to configure the numerical integrator used to simulate the
differential system. These can be adjusted:
\begin{alltt}{\sf
    sub transition \{
      sub ode(atoler = 1.0e-3, rtoler = 1.0e-3, alg = 'rk4') \{
        P <- ode(alpha*P - c*P*Z)
        Z <- ode(e*c*P*Z - m_l*Z)
      \}
    \}
}\end{alltt}

The top-level \hyperref[hyper][transition]{\sf transition}\index{transition}
block, too, has an optional argument, giving the time-step for discrete-time
parts of the system. While not too useful here given the continuous-time
model, this argument can nevertheless be given:
\begin{alltt}{\sf
    sub transition(delta = 1.0) \{
      sub ode(atoler = 1.0e-3, rtoler = 1.0e-3, alg = 'rk4') \{
        P <- ode(alpha*P - c*P*Z)
        Z <- ode(e*c*P*Z - m_l*Z)
      \}
    \}
}\end{alltt}

Arguments, both positional and named, work for blocks as they do for actions.

Finally, named blocks can be used in a \hyperref[hyper][do_then]{\sf
  do.. then}\index{do.. then} clause. The \textsf{do} or \textsf{then} keyword
simply replaces the \textsf{sub} keyword, for example:
\begin{alltt}{\sf
    sub transition(delta = 1.0) \{
      do \{
        alpha ~ normal(mu, sigma)
      \} then ode(atoler = 1.0e-3, rtoler = 1.0e-3, alg = 'rk4') \{
        P <- ode(alpha*P - c*P*Z)
        Z <- ode(e*c*P*Z - m_l*Z - m_q*Z*Z)
      \}
    \}
}\end{alltt}

The implicit wrapping of actions in blocks by GenBi should be intuitive. To
see precisely what transformations are being made to the model, use the
\hyperref[hyper][rewrite]{\sf rewrite}\index{rewrite} command.

\subsection{Expressions\label{Expressions}}

Bi supports expressions over scalars, vectors and matrices, as well as the
special type \textit{element}, which requires some explanation. These are
defined as follows:
\begin{itemize}

\item a \textit{scalar}\index{scalar} is a literal, constant or variable that
  is not declared over any dimensions,

\item a \textit{vector}\index{vector} is a variable declared over
  exactly one dimension,

\item a \textit{matrix}\index{matrix} is a variable declared over exactly two
  dimensions,

\item variables declared over three or more dimensions are not given a special
  name,

\item an \textit{element} is a reference to a single element of any
  variable declared over one or more dimensions.

\end{itemize}

The particular distinction of the element type in Bi will be unfamiliar to
most users. Syntactically, whenever square brackets are used to refer to a
single element of a variable (e.g. \textsf{a[i]}), that symbol is of the
element type.

There are a few implications of these definitions that may not be immediately
obvious (or intuitive):
\begin{itemize}
\item A variable declared over a single dimension of size one is considered a
  vector, not a scalar. Any variable declared over more than one dimension,
  where all of those dimensions have size one, is likewise not considered a
  scalar. The reason for this is that the particular boundary conditions of
  its dimensions may convey different behaviour to that of a scalar.

\item An element, despite referring to a single value, is not a scalar! In
  binary operations, elements may be paired with scalars, but not with vectors
  or matrices. Scalars, on the other hand, may be paired with vectors and
  matrices. The element syntax is, essentially, a concise means of describing
  element-wise transformation kernels (element expressions\index{element
    expressions}) that are readily performed in parallel.

\end{itemize}

Special classes of expression are:
\begin{itemize}
\item A \textit{constant expression}\index{constant expression} is one that
  can be evaluated at compile-time. It must be scalar, and may refer to
  literals, constants and inlines that expand to other constant expressions
  only.
\item A \textit{static expression}\index{static expression} is one that does
  not depend on time. It may refer to literals, constants, variables of type
  \textsf{param}, and inlines that expand to other static expressions only.
\item A \textit{common expression}\index{common expression} is one that does
  not depend on the state of a particular trajectory. It may refer to
  literals, constants, variables of type \textsf{param} or \textsf{input}, and
  inlines that expand to other common expressions only.
\item An \textit{element expression}\index{element expression} is one that
  references at least one element. It may refer to literals, constants,
  scalars and elements only.
\end{itemize}
Note that a constant expression is a static expression, and a static
expression is a common expression.

Element expressions are used for more complicated computations that cannot be
expressed in matrix form, or that when expressed as such become contrived.
They are appropriate when each element of the target variable can be expressed
as a function of scalars and/or the elements of other variables.

Consider the following example describing a Lorenz 96 model:

\begin{alltt}{\sf
    dim x(size = 20, boundary = 'cyclic')
    param F
    state X[x]
    \(\ldots\)
    X[i] <- ode(X[i - 1]*(X[i + 1] - X[i - 2]) - X[i] + F)
}\end{alltt}

Here, \textsf{X} is a vector variable declared along the \textsf{x}
dimension. The final line is an \hyperref[hyper][ode]{\sf ode}\index{ode}
action giving the derivative of the \textsf{i}th element of \textsf{X}. The
presence of the element syntax \textsf{X[i]} is what makes this an element
expression.

The \textit{alias} \textsf{i} on the left-hand side is an arbitrary name for
indexing along the dimension of \textsf{X}. Note the way in which \textsf{i}
is used on the right-hand side to refer to other elements of the \textsf{X}
vector. Indexing expressions that appear between square brackets, such as
\textsf{[i + 1]}, \textsf{[i - 2]} and \textsf{[i]} above must be constant
expressions\index{constant expression}. Note that, in this particular case,
any such expressions that evaluate to outside the bounds of \textsf{X} will
wrap around to the start or end of the vector, as the \textsf{x} dimension is
declared with \textsf{boundary = 'cyclic'}.

Consider another example:
\begin{alltt}{\sf
    dim m(20)
    state D[m,m], x[m], y[m]
    \(\ldots\)
    D[i,j] <- sqrt(pow(x[i] - x[j], 2) - pow(y[i] - y[j], 2))
}\end{alltt}

This computes the Euclidean distance matrix \textsf{D} between several points
with coordinates given in the vectors \textsf{x} and \textsf{y}. This
demonstrates how multiple indices, in this case \textsf{i} and \textsf{j}, may
be used. The particular selection of \textsf{i} and \textsf{j} is again
arbitrary; any name may be used.

Element expressions may use any scalar operators and functions. All operands
and arguments must be scalars or elements. Thus the following is not
permitted:
\begin{alltt}{\sf
    d[i] <- dot(x, y)
}\end{alltt}
as \hyperref[hyper][dot]{\sf dot}\index{dot} is a vector function. One might
consider expressing the same as:
\begin{alltt}{\sf
    do \{
      c <- dot(x,y)
    \} then \{
      d[i] <- c
    \}
}\end{alltt}
where the dot product is first stored in the scalar variable \textsf{c}, and
then copied into each element of the vector \textsf{d}.

\subsection{Operators\label{Operators}}

Arithmetic operators:

{\sf + - * / \% \^{}}

{\sf .+ .- .* ./ .\% .\^{}}

The first set imply matrix operations in the appropriate context. The second
set force element-wise operation.

Bitshift operators:

{\sf << >>}

Comparison operators:

{\sf == != < <= > >= }

Logical operators:

{\sf \&\& ||}

Ternary operators:

{\sf ?:}

\subsection{Functions\label{Functions}}

The following math functions are supported in expressions:

{\sf abs log nanlog exp nanexp max min sqrt pow mod ceil floor gamma lgamma
  sin cos tan asin acos atan atan2 sinh cosh tanh asinh acosh atanh}

\section{Command line interface\label{Command_line_interface}}\index{command line}

Methods are applied to models via the command line interface of Bi. Invoke Bi
via the command line as:
\begin{alltt}{\sf
    bi \textit{command} \textit{options} ...
}\end{alltt}
where \textsf{\textit{command}} is any one of the following:
\begin{itemize}
\item \textsf{\hyperref[hyper][simulate]{simulate}} to simulate trajectories
  from a model,
\item \textsf{\hyperref[hyper][filter]{filter}} for filtering problems using the
  model and observations,
\item \textsf{\hyperref[hyper][smooth]{smooth}} for smoothing problems using the
  model and observations,
\item \textsf{\hyperref[hyper][optimise]{optimise}} for parameter optimisation
  problems using the model and observations,
\item \textsf{\hyperref[hyper][sample]{sample}} for parameter sampling
  problems using the model and observations,
\item \textsf{\hyperref[hyper][draw]{draw}} to visualise a model (useful for
  debugging and development),
\item \textsf{\hyperref[hyper][rewrite]{rewrite}} to process a model and
  output a new model after applying internal transformations and optimisations
  (useful for debugging and development),
\end{itemize}
and available \textsf{\textit{options}} depend on the command.

Options may be specified in a configuration file\index{config file} or on the
command line itself. To use a configuration file, give the name of the file on
the command line, preceded by \textsf{@}, e.g.
\begin{alltt}{\sf
    bi \textit{command} @command.conf
}\end{alltt}

More than one config file may be specified, each preceded by \textsf{@}. An
option given on the command line will override an option of the same name
given in the configuration file.

A config file simply contains a list of command-line options just as they
would be given on the command line itself. For readability, the command-line
options may be spread over any number of lines, and end-of-line comments,
preceded by \textsf{\#}, may appear. The contents of one config file may be
nested in another by using the \textsf{@file.conf} syntax within that
other. This can be useful in avoiding redundancy. For example, the
\textsf{\hyperref[hyper][filter]{filter}} command inherits all the options of
\textsf{\hyperref[hyper][simulate]{simulate}}, and the
\textsf{\hyperref[hyper][sample]{sample}} command inherits all the options of
\textsf{\hyperref[hyper][filter]{filter}}. In this case it may be useful to
write a \textsf{simulate.conf} file that is nested within a
\textsf{filter.conf} file, that is nested within a \textsf{sample.conf} file,
so that the config files inherit options just as the commands do.


\section{Data files\label{Data_files}}

Input data is provided to Bi in the format of NetCDF\index{NetCDF} files. The
structure of such files is specified here.

\subsection{Input files}

Each node in the model is associated with the variable of the same name in the
NetCDF file. Nodes which cannot be associated with a NetCDF variable will
produce warnings at runtime if an attempt is made to read or write to them
while debugging is enabled. Extraneous variables in the NetCDF file are
ignored.

Each variable may be defined along one or more of the following dimensions, in
the order given:
\begin{enumerate}
\item A dimension named \textsf{ns}, used to index multiple experiments set up
  in the same file. If not given for a variable, that variable is assumed to
  be the same for all experiments.
\item Any number of dimensions with names matching those given in the model
  specification.
\item A dimension named \textsf{np}, used to index values of the variable over
  trajectories. If not given for a variable, that variable is assumed to be
  the same for all trajectories. Its length should be greater than or equal to
  the number of trajectories being simulated. Variables of type
  \textsf{param}, \textsf{input} and \textsf{obs} may not use an \textsf{np}
  dimension.
\end{enumerate}

Additionally, a search for \emph{time variables}\index{time variable} is
made. Time variables have a name prefixed by ``time''. Each such variable may
be defined along an arbitrary single inner dimension, and optionally along the
\textsf{ns} dimension as an outer dimension also. The former dimension becomes
a \emph{record dimension}\index{record dimension}. The time variable gives the
time associated with each index of the dimension. Its values must be
monotonically non-decreasing across the dimension.

Any other variables may be specified across the same record dimension,
giving their values at the times given by the associated time variable.
A variable may only be associated with one record dimension, and s- and
p-vars may not be associated with one at all. If a variable is not defined
across a record dimension, it is assumed to have the same value at all
times.

For each time variable, a search is made for an associated \emph{coordinate
  variable}\index{coordinate variable}. This is a variable of the same name,
but with the ``time'' prefix replaced by ``coord''. Its purpose is to specify
the coordinates for spatial nodes of the model that have a corresponding
variable in the file associated with the time variable. Each coordinate
variable should be defined along an optional innermost dimension indexing the
dimensions for nodes of two or more dimensions, then the same record dimension
as the time variable, and optionally along the \textsf{ns} dimension as the
outermost dimension.

If a multidimensional variable is associated with a coordinate variable with
fewer components than the number of dimensions along which it is defined,
these are assumed to index the outermost dimensions. That is, a variable may
be sparse in some dimensions and dense in others, but the sparse dimensions
must be the outermost.

If a variable specified across one or more dimensions in the model cannot be
associated with a coordinate variable, then it is assumed to be dense across
all of its dimensions in the NetCDF file.

Record dimensions, time variables and coordinate variables facilitate sparse
representation by storing only the change-points for each variable over
time. Dense representations are incorporated via the special case where all
variables are associated with the same time variable (or equivalently, where
all time variables are identical), and no coordinate variables are used.

\subsection{Output files}

\section{Bringing it all together\label{Bringing_it_all_together}}

The precise way in which input files and the model specification interact is
best demonstrated in the steps taken to simulate a model. Computing densities
is similar. The \emph{initialisation file} referred to in the proceeding steps
is that given by the {\sf --init-file} command-line option, and the
\emph{input file} that given by {\sf --input-file}.
\begin{enumerate}
\item Any {\sf input} variables in the input file that are not associated with
  a time variable are initialised by reading from the file.
\item The \hyperref[hyper][parameter]{\sf parameter}\index{parameter}
  top-level block is sampled.
\item Any {\sf param} variables in the input file are overwritten by reading
  from the file.
\item Any static expressions extracted by GenBi for precomputation are
  evaluated.
\item The \hyperref[hyper][initial]{\sf initial}\index{initial} top-level
  block is sampled.
\item Any {\sf state} variables in the input file are overwritten by reading
  from the file.
\item The \hyperref[hyper][transition]{\sf transition}\index{transition}
  top-level block is sampled forward through time. Sampling stops at each time
  that an {\sf input} variable is to change, according to the input file, at
  which point the {\sf input} variable is updated and sampling of the
  \hyperref[hyper][transition]{\sf transition}\index{transition} block
  continues. Continuous-time\index{continuous time} components of the
  \hyperref[hyper][transition]{\sf transition}\index{transition} block
  (e.g. \hyperref[hyper][ode]{\sf ode}\index{ode} actions) are always
  actioned. The remaining, discrete-time\index{discrete time} components, are
  only actioned when the current time is an integer multiple of the time step
  given by the \hyperref[hyper][transition]{\sf transition}\index{transition}
  block's {\sf delta} argument.
\end{enumerate}

Note two important points in this procedure:
\begin{itemize}
\item An {\sf input} variable in the input file that is not associated with a
  time variable is initialised before anything else, whereas an {\sf input}
  variable that is associated with a time variable is not initialised until
  simulation begins, even if the first entry of that variable indicates an
  update at time zero.
\item While the \hyperref[hyper][parameter]{\sf parameter}\index{parameter}
  and \hyperref[hyper][initial]{\sf initial}\index{initial} blocks are
  sampled, the samples may be later overwritten from the initialisation
  file. Thus, the initialisation file need not contain a complete set of
  variables in order to ensure that all variables are initialised before
  simulation proceeds. This also ensures
  \hyperref[hyper][Pseudorandom_reproducibility]{pseudorandom
    reproducibility}\index{pseudorandom numbers} regardless of the presence,
  or content, of the initialisation file.
\end{itemize}

\section{Performance guide\label{Performance guide}\index{performance}}

One of the aims of Bi is to alleviate you, the user, from performance
considerations. Consequently, there is little scope in the modelling language
for performance tuning, but little need either. Where you have some control is
in compilation and I/O.

Bi will do the following for you:
\begin{itemize}
\item Precompute constant subexpressions.
\item Precompute static subexpressions in the transition and observation
  models.
\end{itemize}

Reducing redundant or repetitious computations is thus unnecessary where these
are constant or static subexpressions. For example, taking the square-root of
a variance parameter need not be of concern:
\begin{alltt}{\sf
    param sigma2
    \(\ldots\)
    sub transition \{
        epsilon ~ gaussian(mu, sqrt(sigma2))
        \(\ldots\)
    \}
}\end{alltt}
Here, \textsf{sqrt(sigma2)} will be extracted and precomputed outside of the
transition model -- \textsf{sigma2} is a parameter, and so
\textsf{sqrt(sigma2)} a static expression.

Use the \hyperref[hyper][rewrite]{\sf rewrite}\index{rewrite} command if you
wish to inspect the transformations that GenBi makes to a model.

The following I/O and configuration issues are worth considering:
\begin{itemize}
\item When declaring a variable, use the \textsf{io} argument to omit it from
  output files if it will not be of interest.

\item The \textsf{--debug} command-line option is recommended for test runs to
  enable the assertion checking\index{assertion checking} that ensures valid
  code generation and input files. For production runs, however, it need not
  be used, and this will give significant performance improvements.

\item Experiment with the \textsf{--gpu} command-line option to make use of a
  CUDA-enabled GPU\index{GPU}. This will usually improve performance when a
  sufficient number of model trajectories are to be simulated.

\item Experiment with the \textsf{--sse} option to make use of CPU SSE
  instructions\index{SSE}\index{SIMD}. These can provide up to a two-fold
  (double precision) or four-fold (single precision) speed-up. Be careful
  though: models making heavy use of math functions such as \textsf{exp},
  \textsf{log} and trigonometric functions may instead run slower.

\item Experiment with using single precision\index{single precision} by using
  the \textsf{--no-double} command line option. Assuming numerical error is
  tolerable, this can offer significant performance improvements, especially
  when used with the \textsf{--gpu} or \textsf{--sse} options (where single
  precision facilitates both computational and memory efficiencies).

\item \index{multithreading}Experiment with the \textsf{--threads} command
  line option to set the number of CPU threads. One can typically expect a
  linear speedup in the number of threads for methods based on the simulation
  of many trajectories (e.g. the particle filter). Depreciating returns may be
  apparent for other methods.

\item Use optimised libraries for your platform, especially the
  BLAS\index{BLAS} and LAPACK\index{LAPACK} libraries.

\item Use the Intel C++ compiler\index{Intel compiler} if
  available. Anecdotally, this tends to produce code that runs 10-20\% faster
  than \textsf{gcc}\index{gcc}. The \textsf{configure} scripts for both LibBi
  and client programs will automatically detect this compiler\index{compiler},
  and use it if available.

\end{itemize}

\section{Style guide\label{Style_guide}}\index{style guide}

The following naming conventions are recommended:
\begin{itemize}
\item Model names are CamelCase, the first letter always capitalised.
\item Action and block names are all lowercase, with multiple words separated
  by '\_' (the underscore).
\item Dimension and variable names should be consistent, where possible, with
  their counterparts in a description of the model as it might appear in a
  scientific paper. For example, single upper-case letters for the names of
  matrix variables are appropriate, and standard symbols (rather than
  descriptive names) are encouraged. Greek letters should be written out in
  full, the first letter capitalised for the capital version
  (e.g. \textsf{gamma} and \textsf{Gamma}).
\item Comments should be used liberally, with descriptions provided for all
  dimensions and variables in particular. Consider including units as part of
  the description, where relevant.
\item Names ending in '\_' (the underscore) are intended for internal use
  only, and so no such names should generally appear in a model specification.
\item Indent using two spaces, and do not use tabs.
\end{itemize}
