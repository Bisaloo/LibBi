\subsection{Common options\label{Common_options}}

The options in this section are common across all client programs.

Options prefixed with \bitt{--enable} may be negated by replacing this with
\bitt{--disable}. Likewise, options prefixed with \bitt{--with} may be negated
by replacing this with \bitt{--without}.

\subsubsection{General options}

\begin{description}
\item[\bitt{--help}]

Display help.

\item[\bitt{--verbose}] (default off)

Enable verbose reporting.
\end{description}

\subsubsection{Build options}

\begin{description}
\item[\bitt{--dry-gen}] (default off)

Do not (re)generate code.

\item[\bitt{--dry-build}] (default off)

Do not build.

\item[\bitt{--warn}] (default off)

Enable compiler warnings.

\item[\bitt{--native}] (default off)

Compile for native platform.

\item[\bitt{--force}] (default off)

Force all build steps to be performed, even when determined to be not
required. This is useful as a workaround of any faults in detecting
changes, or when system headers or libraries change and recompilation or
relinking is required.

\item[\bitt{--enable-assert}] (default on)

Enable assertion checking (recommended for test runs, not recommended for
production runs).

\item[\bitt{--enable-extradebug}] (default off)

Enable extra debugging options in compilation (recommended in conjunction with
\bitt{--with-gdb}.

\item[\bitt{--enable-diagnostics}] (default off)

Enable diagnostic outputs to standard error.

\item[\bitt{--enable-cuda}] (default off)

Enable CUDA device code.

\item[\bitt{--enable-sse}] (default off)

Enable SSE host code.

\item[\bitt{--enable-mkl}] (default off)

Enable Intel MKL code.

\item[\bitt{--enable-mpi}] (default off)

Enable MPI code.

\item[\bitt{--enable-vampir}] (default off)

Enable Vampir.

\item[\bitt{--enable-single}] (default off)

Use single-precision floating point.
\end{description}

\subsubsection{Run options}

\begin{description}
\item[\bitt{--dry-run}]

Do not run.

\item[\bitt{--seed}] (default 0)

Pseudorandom number generator seed.

\item[\bitt{--threads \textsl{N}}] (default 0)

Run with \bitt{\textsl{N}} threads. If zero, the number of threads used is the
default for OpenMP on the platform.

\item[\bitt{--enable-timing}] (default off)

Enable timings.

\item[\bitt{--enable-output}] (default on)

Enable output.

\item[\bitt{--with-gdb}] (default off)

Run within the \bitt{gdb} debugger.

\item[\bitt{--with-valgrind}] (default off)

Run within \bitt{valgrind}.

\item[\bitt{--with-cuda-gdb}] (default off)

Run within the \bitt{cuda-gdb} debugger.

\item[\bitt{--with-cuda-memcheck}] (default off)

Run within \bitt{cuda-memcheck}.

\item[\bitt{--with-gperftools}] (default off)

Run with \bitt{gperftools} profiler.

\item[\bitt{--gperftools-file}] (default '\textsl{command}.prof')

Output file to use under \bitt{--with-gperftools}.

\item[\bitt{--with-mpi}] (default off)

Run with \bitt{mpirun}.

\item[\bitt{--mpi-np}]

Number of processes when running under \bitt{mpirun}. Corresponds to \bitt{-np}
option to \bitt{mpirun}

\item[\bitt{--mpi-npernode}]

Number of processes per node when running under \bitt{mpirun}. Corresponds to
\bitt{-npernode} option to \bitt{mpirun}.

\end{description}

\subsubsection{Model transformation options}
\begin{description}
\item[\bitt{--transform-extended}] (default off, enabled automatically when
  required)

Linearise the model and symbolically compute Jacobian expressions for use with
the extended Kalman filter.

\item[\bitt{--transform-param-to-state}] (default off)

Augment the state with parameters by interpreting \kwref{param} statements as
\kwref{state}, merging the \blockref{parameter} top-level block into the
\blockref{initial} top-level block, and merging the
\blockref{proposal_parameter} top-level block into the
\blockref{proposal_initial} top-level block.

This is useful for joint state and parameter estimation using filters.

\item[\bitt{--transform-obs-to-state}] (default off)

Augment the state with observations by interpreting \kwref{obs} statements as
\kwref{state}, merging the \kwref{observation} top-level block into the
\blockref{transition}, \blockref{initial} and \blockref{proposal_initial}
top-level blocks, and merging the \blockref{lookahead_observation} top-level
block into the \blockref{lookahead_transition} top-level block.

This is useful for producing simulated data sets from a model.

\item[\bitt{--transform-initial-to-param}] (default off)

Augment the parameters with the initial conditions of the state variables by
adding a new \kwref{param} for each \kwref{state} variable to hold its initial
value, merging the \blockref{initial} top-level block into the
\blockref{parameter} top-level block, merging the \blockref{proposal_initial}
top-level block into the \blockref{proposal_parameter} top-level block, and
replacing both the \blockref{proposal_parameter} and
\blockref{proposal_initial} top-level blocks with $\delta$-masses.

This is useful for treating the initial conditions as parameters in sampling
schemes such as PMCMC and SMC$^2$. Note that the transformation is semantic,
not substantial.

\end{description}
