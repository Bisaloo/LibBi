\subsection{\bitt{model}\label{model}}

Declare a model.

\subsubsection*{Synopsis\label{model_Synopsis}}
\begin{bicode}
    model \textsl{Name} \{
      \(\ldots\)
    \}
\end{bicode}

\subsubsection*{Description\label{model_Description}}

A \bitt{model} statement is the outermost statement of a model specification,
declaring and naming the model.

Immediately within the \bitt{model} statement, the following named top-level blocks should
be used to specify the model:
\begin{itemize}
\item \blockref{parameter}, specifying the prior density over parameters,
\item \blockref{initial}, specifying the prior density over initial conditions,
\item \blockref{transition}, specifying the transition density, and
\item \blockref{observation}, specifying the observation density.
\end{itemize}

The following named top-level blocks are optional, but may be required in order to use particular methods:
\begin{itemize}
\item \blockref{proposal_parameter}, specifying the proposal density over parameters,
\item \blockref{proposal_initial}, specifying the proposal density over initial conditions,
\item \blockref{lookahead_transition}, specifying a lookahead density to accompany the transition density, and
\item \blockref{lookahead_observation}, specifying a lookahead density to accompany the observation density.
\end{itemize}

\subsection{\bitt{dim}\label{dim}\index{dim}}

Declare a dimension.

\subsubsection*{Synopsis\label{dim_Synopsis}}
\begin{bicode}
    dim \textsl{name}(100, 'cyclic')
    dim \textsl{name}(size = 100, boundary = 'cyclic')
\end{bicode}


\subsubsection*{Description\label{dim_Description}}

A \bitt{dim} statement declares a dimension with a given size and boundary
condition.

A \bitt{dim} statement may only be used at the top level of the model
specification. Dimensions must be declared before any variables are declared
along them.

\subsubsection*{Arguments\label{dim_Arguments}}

\begin{description}
\item[\bitt{size}] (position 0, mandatory)

Length of the dimension.

\item[\bitt{boundary}] (position 1, default 'none')

Boundary condition of the dimension. Valid values are:

\begin{description}
\item[\bitt{'none'}]

No boundary condition.

\item[\bitt{'cyclic'}]

Cyclic boundary condition; all indices are taken modulo the \bitt{size} of
the dimension.
\end{description}
\end{description}

\subsection{\bitt{input}\index{input}\label{input}, \bitt{noise}\index{noise}\label{noise}, \bitt{obs}\index{obs}\label{obs}, \bitt{param}\index{param}\label{param} and \bitt{state}\index{state}\label{state}}

Declare an input, noise, observed, parameter or state variable.

\subsubsection*{Synopsis\label{var_Synopsis}}

\begin{bicode}
    state x
    state x[i]
    state x(has_output = 0)
    state x[i](has_output = 0)
\end{bicode}

\subsubsection*{Description\label{var_Description}}

Any of these statements declare a variable of the given type, along the list
of dimensions given between the square brackets.

These statements may only be used at the top level of the model
specification. Dimensions along which a variable extends must be declared
prior to the declaration of the variable.

\subsubsection*{Arguments\label{var_Arguments}}

\begin{description}
\item[\bitt{has\_input}] (default 1)\index{I/O}

Include variable when doing input from a file?

\item[\bitt{has\_output}] (default 1)\index{I/O}

Include variable when doing output to a file?

\item[\bitt{input\_name}] (default the same as the name of the
  variable)\index{I/O}

Name to use for the variable in input files.

\item[\bitt{output\_name}] (default the same as the name of the
  variable)\index{I/O}

Name to use for the variable in output files.
\end{description}

\subsection{\bitt{const}\label{const}\index{const}}

Declare a constant\index{constant}.

\subsubsection*{Synopsis\label{const_Synopsis}}
\begin{bicode}
    const \textsl{name} = \textsl{constant_expression}
\end{bicode}

\subsubsection*{Description\label{const_description}}

A \bitt{const}\index{const} statement declares a constant\index{constant}, the
value of which is evaluated using the given constant
expression\index{constant\,expression}. The constant may then be used, by
name, in other expressions.

A constant\index{constant} may be declared anywhere in a model specification,
but always has global scope\index{scope}.

\subsection{\bitt{inline}\label{inline}\index{inline}}

Declare an inline expression\index{inline expression}.

\subsubsection*{Synopsis\label{inline_synopsis}}
\begin{bicode}
    inline \textsl{name} = \textsl{expression}
\end{bicode}

\subsubsection*{Description\label{inline_description}}

An \bitt{inline}\index{inline} statement declares an inline
expression\index{inline expression}. The inline may then be used, by name, in
other expressions, as long as it will not violate any constraints on those
expressions (e.g. an inline expression named within a constant expression must
itself be a constant expression).

An inline expression\index{inline expression} may be declared anywhere in a
model specification, but always has global scope\index{scope}.
